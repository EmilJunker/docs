\documentclass[paper=a4]{scrartcl}
\usepackage[utf8]{inputenc}
\usepackage[T1]{fontenc}
\usepackage{url}
\usepackage{enumitem}

\title{Overview of IDS Internals}
\author{Rolf Krahl}

% Add a revision hint as a unnumbered footnote
\newcommand{\revhint}{%
  \begingroup%
  \let\thefootnote\relax%
  \footnote{Revision: \input{.revision}}%
  \addtocounter{footnote}{-1}%
  \endgroup%
}

\begin{document}

\maketitle

\section{Introduction}

\revhint{}%
The ICAT Data Service (IDS) operates with delayed actions, internal
states, and multiple threads.  This makes it somewhat difficult to
understand from mere reading the sources what actions may be performed
under certain conditions.  The knowledge of the internal processes in
IDS may be needed to properly implement a storage plugin in a
non-trivial setup.  In particular, one might need to know the context
that each of the plugin's methods may be called in.  This text shall
provide a reference to accommodate this need.  It is however only an
overview.  In order to keep the representation clear, many details
have been omitted.  Furthermore it is somewhat biased towards the
point of view of a storage plugin developer and it is restricted to
the case that the IDS is configured as a two level storage.  The text
is based on IDS server 1.10.0 and IDS plugin 1.5.0.

When working with data objects in the storage, IDS can be configured
to use either one of two different storage units: datasets or
datafiles.  This text tries to describe the behavior in both cases.

The workflow in the IDS can be sketched as follows: the IDS waits for
incoming service requests coming from the user.  In some cases, these
requests may be completed immediately.  In other cases, a deferred
operation that will be processed in the background later is queued.

Before describing this workflow in detail, we give the definition of
the status of a data object in Section \ref{sec:states}.  Section
\ref{sec:defops} describes the different deferred operations and what
is being done for each of them.  The various operations that are
queued for any given data object are kept track of in a finite state
machine.  This is detailed in Section \ref{sec:fsm}.  Section
\ref{sec:requests} lists the available service request.  Besides
processing the service requests, there are some maintenance tasks
running regularly in the background.  These tasks are described in
Section \ref{sec:maintenance}.  Finally, we provide an index of the
context that each storage plugin method is called from in Section
\ref{sec:plugincalls}.


\section{Internal states}
\label{sec:states}

The processing of a data object is influenced by its internal state
and what operations have been queued for them.  We say that an
operation is \emph{in process} on a data object if this operation is
either in the queue or if it is currently being executed for that
data object.

The status of a data object may either be \texttt{ONLINE},
\texttt{ARCHIVED}, or \texttt{RESTORING}.  In this context,
\texttt{ARCHIVED} refers to files being available only in the archive
storage, while \texttt{ONLINE} refers to the files being available in
the main storage as well.  The \texttt{RESTORING} state describes the
process of making files from the archive storage available in the
main storage.

A data object is \texttt{RESTORING} if a \texttt{RESTORE}
operation on that data object is in process.

If this is not the case, the status is \texttt{ARCHIVED}
\begin{itemize}[noitemsep,topsep=0pt]
\item if any operation other than \texttt{WRITE} is in process
for the data object, or
\item for a datafile, if it does not exist in the main storage, or
\item for a dataset, if it is not empty (e.g. related datafiles
exist), but the dataset directory does not exist in the main storage.
\end{itemize}

In all other cases, the status is \texttt{ONLINE}.


\section{Deferred operations}
\label{sec:defops}

When the queue of deferred operations is processed, a new thread is
started to execute each of them.  This section describes what these
threads do in each case.

\subsection{\texttt{ARCHIVE}}
\label{sec:defops:archive}

Delete the data object from the main storage.  This leaves only the
files in the archive storage.

\subsection{\texttt{RESTORE}}
\label{sec:defops:restore}

Make the data object from the archive storage available in the
main storage:
\begin{itemize}[noitemsep,topsep=0pt]
\item For a datafile, get the file from the archive storage and
copy it to the main storage.
\item For a dataset, get the ZIP file of the dataset from the
archive storage, extract it, and store the contained datafiles in a
dataset directory in the main storage.
\end{itemize}

\subsection{\texttt{WRITE}}
\label{sec:defops:write}

Overwrite the file in the archive storage with the latest version
from the main storage:
\begin{itemize}[noitemsep,topsep=0pt]
\item For a datafile, get the file from the main storage and copy it
to the archive storage.
\item For a dataset, if the dataset directory does not exist in the
main storage, delete the dataset from the archive storage. Otherwise,
get all datafiles that belong to the dataset (according to ICAT) from
the main storage, package them as a ZIP file, and store it in the
archive storage.
\end{itemize}

\subsection{\texttt{DELETE}}
\label{sec:defops:delete}

This operation is only available for datafiles, not for datasets.

Delete the datafile from the archive storage.


\section{Finite state machine}
\label{sec:fsm}

The finite state machine manages the queue of deferred operations for
the data objects.  For each data object, only one operation can be
queued at a time.  Thus, the finite state machine sometimes uses the
\texttt{WRITE\_THEN\_ARCHIVE} state, to remember to queue up an
\texttt{ARCHIVE} deferred operation after the \texttt{WRITE} deferred
operation is done.  The queued operation of a data object may be
changed if another operation is queued for it while the previous one
is still waiting in the queue, see Tab.\ \ref{tab:fsm} for the update
matrix.

\begin{table}
 \begin{tabular}{l|llll}
  for datafiles                 & \texttt{ARCHIVE} & \texttt{RESTORE} & \texttt{WRITE} & \texttt{DELETE} \\
  \hline
  none                          & \texttt{ARCHIVE} & \texttt{RESTORE} & \texttt{WRITE} & \texttt{DELETE} \\
  \texttt{ARCHIVE}              & \texttt{ARCHIVE} & none & \texttt{ARCHIVE} & \texttt{DELETE} \\
  \texttt{RESTORE}              & \texttt{ARCHIVE} & \texttt{RESTORE} & \texttt{RESTORE} & \texttt{DELETE} \\
  \texttt{WRITE}                & \texttt{WRITE\_THEN\_ARCHIVE} & \texttt{WRITE} & \texttt{WRITE} & none \\
  \texttt{DELETE}               & \texttt{DELETE} & \texttt{DELETE} & \texttt{DELETE} & \texttt{DELETE} \\
  \texttt{WRITE\_THEN\_ARCHIVE} & \texttt{WRITE\_THEN\_ARCHIVE} & \texttt{WRITE} & \texttt{WRITE\_THEN\_ARCHIVE} & none \\
 \end{tabular}
 \bigbreak
 \begin{tabular}{l|lll}
  for datasets                  & \texttt{ARCHIVE} & \texttt{RESTORE} & \texttt{WRITE} \\
  \hline
  none                          & \texttt{ARCHIVE} & \texttt{RESTORE} & \texttt{WRITE} \\
  \texttt{ARCHIVE}              & \texttt{ARCHIVE} & \texttt{RESTORE} & \texttt{WRITE\_THEN\_ARCHIVE} \\
  \texttt{RESTORE}              & \texttt{ARCHIVE} & \texttt{RESTORE} & \texttt{WRITE} \\
  \texttt{WRITE}                & \texttt{WRITE\_THEN\_ARCHIVE} & \texttt{WRITE} & \texttt{WRITE} \\
  \texttt{WRITE\_THEN\_ARCHIVE} & \texttt{WRITE\_THEN\_ARCHIVE} & \texttt{WRITE} & \texttt{WRITE\_THEN\_ARCHIVE} \\
 \end{tabular}
 \caption{Update matrix for queued operations.  Matrix entries are
  the result if a new operation (column) is queued while a previous
  one (row) is still waiting in the queue.}
 \label{tab:fsm}
\end{table}

The queue is processed regularly by a timer task that starts a new
thread to execute each of the pending operations.

\texttt{WRITE} operations for datasets are processed with a delay
that is fixed when the operation is queued.  Any subsequent
\texttt{WRITE} operation queued for the same dataset pushes this
delay further.


\section{Service requests}
\label{sec:requests}

In this section, we consider each IDS request and describe in detail
what is done in each case.  We consider only requests that interact in
any way with the storage plugin.  The requests \texttt{getApiVersion},
\texttt{getIcatUrl}, \texttt{getServiceStatus}, \texttt{getSize},
\texttt{isReadOnly}, \texttt{isTwoLevel}, \texttt{ping}, and
\texttt{version} are thus skipped.

Several service calls that deal with a selection of data objects in
the storage expect lists of datafile, dataset, and investigation ids
as parameter.  As most internal processing is done at the level of the
configured storage unit, it only matters which datafiles or datasets,
respectively, are concerned.  In these cases, we use the terms
\emph{selected datafiles} and \emph{selected datasets} regardless
whether the data objects have actually been selected by datafile,
dataset, or investigation id.

Another way to select data objects is to specify a
\emph{prepared id}.  This can be done only after making a
\texttt{prepareData} service call
(Sec.~\ref{sec:requests:prepareData}), and it automatically
selects all data objects that where previously prepared.

\subsection{\texttt{archive}}

Queue an \texttt{ARCHIVE} deferred operation for each of the selected
data objects.

\subsection{\texttt{delete}}
\label{sec:requests:delete}

If the storage unit is set to dataset and any of the selected
datasets is not \texttt{ONLINE} (Sec.~\ref{sec:states}), queue a
\texttt{RESTORE} deferred operation for them and throw a
\texttt{DataNotOnlineException}.

Otherwise delete the selected data objects from ICAT and from the main
storage.  Finally, queue a \texttt{DELETE} deferred operation for each
of the selected datafiles, or a \texttt{WRITE} deferred operation for
each of the selected datasets, depending on the storage unit, in order
to get the files deleted from the archive storage as well.

\subsection{\texttt{getData}}
\label{sec:requests:getdata}

If any of the selected data objects is not \texttt{ONLINE}
(Sec.~\ref{sec:states}), queue a \texttt{RESTORE} deferred operation
for them and throw a \texttt{DataNotOnlineException}.

Otherwise get the selected data objects from the main storage and
stream their content to the client.

\subsection{\texttt{getDatafileIds}}

Return a list of all datafile ids that are part of the set of
selected data objects.

\subsection{\texttt{getLink}}
\label{sec:requests:getlink}

If the selected datafile is not \texttt{ONLINE}
(Sec.~\ref{sec:states}), queue a \texttt{RESTORE} deferred operation
for it and throw a \texttt{DataNotOnlineException}.

Otherwise get the path of the file from main storage, set an ACL to
grant read permission to the user on the file, and return a link to
the file.

\subsection{\texttt{getStatus}}

Return \texttt{ONLINE} if all selected data objects are
\texttt{ONLINE} (Sec.~\ref{sec:states}).  Else, if at least one data
object is archived, return \texttt{ARCHIVED}, otherwise return
\texttt{RESTORING}.

\subsection{\texttt{isPrepared}}

If any of the selected data objects from a previously prepared
data selection (Sec.~\ref{sec:requests:prepareData}) is not yet
\texttt{ONLINE} (Sec.~\ref{sec:states}), queue a \texttt{RESTORE}
deferred operation for them and return \texttt{false}.

Otherwise return \texttt{true}.

\subsection{\texttt{prepareData}}
\label{sec:requests:prepareData}

If any of the selected data objects is not \texttt{ONLINE}
(Sec.~\ref{sec:states}), queue a \texttt{RESTORE} deferred operation
for them.  Store the data selection and other parameter of the request
and return the \emph{prepared id} for later referral.

\subsection{\texttt{put}}
\label{sec:requests:put}

If the storage unit is set to dataset and the referenced dataset is
not \texttt{ONLINE} (Sec.~\ref{sec:states}), queue a \texttt{RESTORE}
deferred operation for it and throw a \texttt{DataNotOnlineException}.

Otherwise store the uploaded datafile in the main storage, create the
datafile object in ICAT, and queue a \texttt{WRITE} deferred operation
for the respective data object.

\subsection{\texttt{reset}}

Reset the status flag of the selected data objects so that they can
be queried again after a failed \texttt{restore} operation.

\subsection{\texttt{restore}}

Queue a \texttt{RESTORE} deferred operation for each of the selected
data objects.

\subsection{\texttt{write}}

Queue a \texttt{WRITE} deferred operation for each of the selected
data objects.


\section{Maintenance tasks}
\label{sec:maintenance}

There are some maintenance tasks running in the background
independently from user actions.

\subsection{\texttt{FileChecker}}
\label{sec:maintenance:filechecker}

The File Checker is only available for the storage unit dataset.

Iterate over all datasets in the ICAT including their datafiles.  For
each dataset, get the ZIP file from the archive storage and inspect it
to check that the list of datafiles in the ZIP file matches the list
of datafiles in ICAT and that length and checksum matches for each of
the datafiles.

\subsection{\texttt{Tidier}}
\label{sec:maintenance:tidier}

Query the main storage plugin for a list of data objects to remove in
order to get the overall size of the main storage below the configured
limit.  Queue an \texttt{ARCHIVE} deferred operation for each of the
returned data objects.


\section{Index of plugin method calls}
\label{sec:plugincalls}

In this section, we list for each of the storage plugin methods the
context that it is called from.

\subsection{\texttt{archiveStorage.delete(DsInfo)}}

Called from the \texttt{DsWriter} (Sec.~\ref{sec:defops:write}) if
the dataset directory does not exist in the main storage.

\subsection{\texttt{archiveStorage.delete(String)}}

Called from the \texttt{DfDeleter} (Sec.~\ref{sec:defops:delete}) to
delete the datafile from the archive storage.

\subsection{\texttt{archiveStorage.get(DsInfo, Path)}}

Called from the \texttt{DsRestorer} (Sec.~\ref{sec:defops:restore})
to extract the dataset's ZIP file from the archive storage into the
main storage.

Called from the \texttt{FileChecker}
(Sec.~\ref{sec:maintenance:filechecker}) to inspect the ZIP file.

\subsection{\texttt{archiveStorage.put(DsInfo, InputStream)}}

Called from the \texttt{DsWriter} (Sec.~\ref{sec:defops:write}) to
store the ZIP file in the archive storage.

\subsection{\texttt{archiveStorage.put(InputStream, String)}}

Called from the \texttt{DfWriter} (Sec.~\ref{sec:defops:write}) to
store the file in the archive storage.

\subsection{\texttt{archiveStorage.restore(MainStorageInterface, DfInfos)}}

Called from the \texttt{DfRestorer} (Sec.~\ref{sec:defops:restore})
to copy the datafile from the archive storage to the main storage.

\subsection{\texttt{mainStorage.delete(DsInfo)}}

Called from the \texttt{DsArchiver} (Sec.~\ref{sec:defops:archive})
to delete the dataset from the main storage.

\subsection{\texttt{mainStorage.delete(String, String, String)}}

Called from the \texttt{DfArchiver} (Sec.~\ref{sec:defops:archive})
to delete the datafile from the main storage.

Called from \texttt{IdsBean} while processing a \texttt{delete}
request (Sec.~\ref{sec:requests:delete}) to delete the selected
datafiles from the main storage.

Called from \texttt{IdsBean} while processing a \texttt{put} request
(Sec.~\ref{sec:requests:put}) to delete the uploaded file from the
main storage again in the case that an error occurred while creating
the corresponding datafile object in ICAT.

\subsection{\texttt{mainStorage.exists(DsInfo)}}

Called from \texttt{IdsBean} while processing various service requests
(Sec.~\ref{sec:requests}) to determine the status of a dataset.

Called from the \texttt{DsWriter} (Sec.~\ref{sec:defops:write}) and
the \texttt{DsRestorer} (Sec.~\ref{sec:defops:restore}) to check if
the dataset directory exists.

\subsection{\texttt{mainStorage.exists(String)}}

Called from the \texttt{DfRestorer} (Sec.~\ref{sec:defops:restore})
to check if the datafile exists.

\subsection{\texttt{mainStorage.get(String, String, String)}}

Called from the \texttt{DsWriter} (Sec.~\ref{sec:defops:write}) to
get the datafiles and add them to the ZIP archive.

Called from the \texttt{DfWriter} (Sec.~\ref{sec:defops:write}) to
get the datafile and copy it to the archive storage.

Called from \texttt{IdsBean} while writing data into a ZIP file.

\subsection{\texttt{mainStorage.getDatafilesToArchive(long, long)}}

Called from the \texttt{Tidier} (Sec.~\ref{sec:maintenance:tidier}) to
query the main storage plugin for a list of datafiles to remove.

\subsection{\texttt{mainStorage.getDatasetsToArchive(long, long)}}

Called from the \texttt{Tidier} (Sec.~\ref{sec:maintenance:tidier}) to
query the main storage plugin for a list of datasets to remove.

\subsection{\texttt{mainStorage.getPath(String, String, String)}}

Called from \texttt{IdsBean} while processing a \texttt{getLink}
request (Sec.~\ref{sec:requests:getlink}) to get the path of the file
from main storage.

\subsection{\texttt{mainStorage.put(DsInfo, String, InputStream)}}

Called from \texttt{IdsBean} while processing a \texttt{put} request
(Sec.~\ref{sec:requests:put}) to store the uploaded file in the main
storage.

\subsection{\texttt{mainStorage.put(InputStream, String)}}

Called from the \texttt{DsRestorer} (Sec.~\ref{sec:defops:restore})
to store the files extracted from a ZIP archive in the main storage.


\subsection{\texttt{mainStorage.lock(DsInfo, boolean)}}

Called from the \texttt{LockManager} to acquire a lock on the file
system level for a data object in the main storage.


\end{document}
